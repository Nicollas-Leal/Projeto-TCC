\chapter{Introdução}

\section{Conceituação}

O avanço das tecnologias digitais transformou profundamente a maneira como indivíduos interagem com informações, serviços e sistemas computacionais. Contudo, apesar do acesso à internet e a dispositivos eletrônicos estar cada vez mais difundido, ainda há uma parcela significativa da população que enfrenta dificuldades para utilizar ferramentas digitais básicas. Essa realidade está associada à \textbf{baixa alfabetização informática}, entendida como a limitação de conhecimentos e habilidades necessárias para operar computadores, navegadores, aplicativos e outros recursos digitais de forma autônoma.

Tal dificuldade impacta diretamente o processo de inclusão digital, uma vez que pessoas com baixo domínio tecnológico acabam excluídas de serviços essenciais e oportunidades que dependem do uso de plataformas digitais. Nesse contexto, desenvolver softwares e sites que considerem as necessidades e limitações desse público é um desafio tanto técnico quanto social.

O campo da \textbf{Interação Humano-Computador (IHC)} oferece importantes contribuições nesse sentido, pois se dedica ao estudo da relação entre usuários e sistemas, buscando promover \textbf{interfaces acessíveis, intuitivas e eficientes}. Aliar conceitos de IHC a \textbf{métricas de usabilidade e acessibilidade} torna possível avaliar e aprimorar soluções de modo sistemático, garantindo que sejam realmente eficazes para pessoas com pouca familiaridade com a tecnologia.

Assim, este trabalho propõe-se a investigar e organizar diretrizes e métricas que possam servir de base para o desenvolvimento de sistemas voltados a pessoas com baixa alfabetização informática. Desse modo, o estudo pretende responder à seguinte \textbf{questão de pesquisa}: \textbf{Como criar um site que possa ser facilmente utilizado por pessoas que nunca tiveram contato ou possuem pouco conhecimento em informática, garantindo uma interação intuitiva e acessível?}

\section{Objetivos}

O objetivo geral deste estudo é \textbf{propor um guia de desenvolvimento de software voltado a pessoas com baixa alfabetização informática}, reunindo métricas, boas práticas e componentes de interface que contribuam para o desenvolvimento de sistemas acessíveis e intuitivos.

Para alcançar esse propósito, são estabelecidos os seguintes objetivos específicos:

\begin{itemize}
    \item Identificar, por meio da revisão de literatura, as principais métricas de usabilidade e acessibilidade aplicáveis ao público com baixa proficiência digital;
    \item Analisar elementos de interface e design de interação que possam ser adaptados para facilitar o uso por pessoas com pouca experiência tecnológica;
    \item Avaliar como métricas de IHC podem orientar a criação de interfaces mais intuitivas e inclusivas;
    \item Estruturar um catálogo de componentes e bibliotecas de apoio que sirva como referência prática para o desenvolvimento de softwares destinados a esse público.
\end{itemize}

\section{Metodologia}

Para atingir os objetivos propostos, este estudo adota uma metodologia de caráter \textbf{exploratório e bibliográfico}, com base em princípios da \textbf{Design Science Research (DSR)}, conforme descrito por \citet{dresch2014design}. A DSR é adequada a pesquisas que buscam gerar artefatos --- no caso deste trabalho, um guia de desenvolvimento e um catálogo de componentes --- a partir da identificação de um problema real e da sistematização de evidências científicas e técnicas.

O estudo parte da \textbf{revisão da literatura} sobre alfabetização digital, IHC, usabilidade, acessibilidade e design centrado no usuário, buscando identificar lacunas e boas práticas aplicáveis ao público-alvo. Em seguida, realiza-se a \textbf{análise e categorização de métricas e diretrizes} encontradas, relacionando-as às dificuldades enfrentadas por pessoas com baixa familiaridade tecnológica.

A partir desses resultados, será estruturado um \textbf{modelo conceitual de guia}, reunindo métricas, componentes de interface e recomendações práticas. Por fim, o estudo propõe direções para testes e validações futuras desses elementos em contextos reais de aplicação.

\section{Organização do texto}

Este artigo está estruturado em cinco seções. A \textbf{primeira seção} apresenta a introdução, abordando a contextualização do problema, a questão de pesquisa, os objetivos, a metodologia e a estrutura geral do trabalho. A \textbf{segunda seção} traz a fundamentação teórica, que discute os conceitos de alfabetização digital, IHC, usabilidade, acessibilidade e design centrado no usuário. Na \textbf{terceira seção}, descreve-se a metodologia adotada, detalhando as etapas de levantamento, análise e sistematização das informações. A \textbf{quarta seção} apresenta os resultados obtidos, incluindo a síntese das métricas, diretrizes e componentes que compõem o guia de desenvolvimento. Por fim, a \textbf{quinta seção} reúne as considerações finais, destacando as contribuições do estudo e as perspectivas para trabalhos futuros.
