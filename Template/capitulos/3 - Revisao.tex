\chapter{Revisão da Literatura / Trabalhos Relacionados}
\label{sec:revisao}

Este capítulo apresenta a revisão da literatura realizada com o objetivo de identificar e analisar trabalhos relacionados ao desenvolvimento e à avaliação de interfaces voltadas a usuários com baixa alfabetização informática. A revisão busca situar o presente trabalho no contexto do estado da arte, evidenciando abordagens existentes, métodos utilizados e lacunas ainda não plenamente exploradas pela literatura.

A revisão foi conduzida de forma estruturada, seguindo os princípios de uma \textit{Rapid Review}, permitindo uma análise sistemática e objetiva dos estudos mais relevantes para o tema investigado.

\section{Metodologia da Revisão da Literatura}

A revisão da literatura foi conduzida por meio de uma \textit{Rapid Review}, abordagem que se caracteriza por aplicar procedimentos sistemáticos de seleção e análise de estudos, porém de maneira mais ágil quando comparada a revisões sistemáticas tradicionais. Essa estratégia mostrou-se adequada ao contexto deste trabalho, pois possibilita identificar evidências relevantes sem comprometer o rigor metodológico.

A busca pelos estudos foi realizada na base de dados \textbf{Scopus}, escolhida por sua ampla cobertura de publicações científicas nas áreas de computação, Interação Humano-Computador e design de interfaces. Os resultados obtidos foram importados e organizados na ferramenta \textbf{Parsifal}, utilizada para apoiar o processo de triagem, extração de dados e avaliação da qualidade dos artigos selecionados.

Os critérios de inclusão consideraram estudos que abordassem, ao menos, um dos seguintes aspectos: métricas de usabilidade e acessibilidade, diretrizes de design de interface, métodos de avaliação em IHC ou estratégias voltadas a usuários com baixa proficiência digital. Foram excluídos trabalhos que não apresentavam relação direta com interfaces digitais, que não envolviam usuários finais ou que não forneciam evidências empíricas ou conceituais relevantes para o tema.

\section{Caracterização dos Estudos Selecionados}

Os estudos selecionados abrangem diferentes contextos de aplicação, incluindo sistemas governamentais, aplicações de saúde, plataformas web e tecnologias assistivas. Observa-se uma predominância de pesquisas voltadas a públicos considerados vulneráveis do ponto de vista digital, como pessoas idosas, usuários com comprometimentos cognitivos leves e indivíduos com pouca experiência no uso de computadores e dispositivos digitais.

Em relação aos métodos empregados, os trabalhos analisados utilizam predominantemente testes de usabilidade com usuários reais, avaliações heurísticas, análises baseadas em métricas quantitativas (como tempo de tarefa e taxa de erro) e abordagens qualitativas, como entrevistas e observação direta. Essa diversidade metodológica evidencia a complexidade do problema e a necessidade de múltiplas perspectivas para compreender as dificuldades enfrentadas por usuários com baixa alfabetização informática.

\section{Síntese dos Trabalhos Relacionados}

A análise dos estudos permite identificar quatro eixos principais de contribuição na literatura. O primeiro eixo concentra trabalhos que investigam métricas de usabilidade e acessibilidade, destacando indicadores como eficácia, eficiência, satisfação e carga cognitiva como fundamentais para avaliar interfaces utilizadas por usuários iniciantes. Esses estudos demonstram que métricas tradicionais de IHC permanecem relevantes, mas precisam ser interpretadas considerando o perfil do usuário.

O segundo eixo reúne pesquisas focadas em diretrizes e princípios de design de interface. Esses trabalhos ressaltam a importância da simplicidade visual, da consistência na navegação, do uso de linguagem clara e da redução da sobrecarga cognitiva. Interfaces que seguem essas diretrizes tendem a apresentar melhores resultados de desempenho e menor taxa de erro entre usuários com baixa proficiência digital.

O terceiro eixo engloba estudos que exploram métodos e estratégias de avaliação, como avaliações heurísticas adaptadas, testes iterativos e uso combinado de dados quantitativos e qualitativos. Os resultados indicam que abordagens iterativas e centradas no usuário são essenciais para identificar problemas de usabilidade que não seriam detectados apenas por inspeções técnicas.

Por fim, um quarto eixo contempla trabalhos voltados ao desenvolvimento de sistemas específicos, como aplicações de saúde e tecnologias assistivas, que demonstram como a adaptação cultural, linguística e cognitiva da interface impacta diretamente a aceitação e o uso efetivo das soluções propostas.

\section{Comparação com o Trabalho Proposto}

Embora os trabalhos analisados apresentem contribuições relevantes, observa-se que a maioria deles aborda métricas, diretrizes ou componentes de interface de forma isolada, geralmente aplicada a contextos específicos. Poucos estudos propõem uma visão integrada que reúna métricas de avaliação, princípios de design e componentes reutilizáveis voltados explicitamente a desenvolvedores que não possuem formação em Interação Humano-Computador.

Nesse contexto, o presente trabalho se diferencia ao propor a consolidação desses elementos em um guia prático de desenvolvimento, com foco em usuários com baixa alfabetização informática. Ao articular resultados da literatura com uma abordagem metodológica estruturada, busca-se oferecer uma referência aplicável e acessível para o desenvolvimento de interfaces mais intuitivas e inclusivas.

\section{Considerações Finais do Capítulo}

A revisão da literatura evidencia avanços significativos no estudo da usabilidade e da acessibilidade para públicos com baixa proficiência digital, mas também revela lacunas relacionadas à integração prática desses conhecimentos. Os trabalhos analisados reforçam a necessidade de abordagens centradas no usuário e fundamentadas em evidências empíricas.

Os resultados desta revisão servem como base para a proposta apresentada nos capítulos seguintes, orientando a definição das métricas, dos componentes de interface e da metodologia adotada para o desenvolvimento do guia proposto.
