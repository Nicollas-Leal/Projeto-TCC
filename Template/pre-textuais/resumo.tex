\dresumo{
O presente trabalho propõe o desenvolvimento de um guia de apoio ao design e à implementação de sistemas voltados a pessoas com baixa alfabetização informática. O estudo parte do reconhecimento de que a falta de familiaridade com ferramentas digitais representa uma barreira significativa à inclusão tecnológica e social. A pesquisa busca identificar métricas, princípios e componentes de interface que possam orientar a criação de softwares e sites mais acessíveis, intuitivos e eficazes para esse público. A metodologia adotada é de caráter exploratório e bibliográfico, fundamentada nos princípios da Design Science Research, a fim de construir um artefato conceitual — um guia e um catálogo de componentes — que reúna boas práticas de Interação Humano-Computador (IHC), usabilidade e acessibilidade. Espera-se que os resultados contribuam para o avanço do design inclusivo e sirvam de referência a desenvolvedores e pesquisadores interessados na criação de tecnologias mais acessíveis e democráticas.
}
{Interação Humano-Computador; Usabilidade; Acessibilidade; Alfabetização Informática; Design Inclusivo}
